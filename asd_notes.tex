\documentclass[10pt, oneside]{article} 
\usepackage{amsmath, amsthm, amssymb, calrsfs, wasysym, verbatim, bbm, color, graphics, geometry}
\usepackage{polski}
\usepackage[utf8]{inputenc}
\usepackage{minted}
\usepackage{algorithm}
\usepackage{algorithmicx}
\usepackage[noend]{algpseudocode}
\usepackage{url}
\usepackage{tikz}
\usepackage{pgfplots}

\usetikzlibrary{matrix,arrows,automata}

\geometry{tmargin=.75in, bmargin=.75in, lmargin=.75in, rmargin = .75in}  

\theoremstyle{remark}
\newtheorem*{example}{Przykład}


% Cormen's cost analysis
\newcommand{\TITLE}[1]{\item[#1]}
\renewcommand{\algorithmiccomment}[1]{$/\!/$ \parbox[t]{4.5cm}{\raggedright #1}}
% ugly hack for for/while
\newbox\fixbox
\renewcommand{\algorithmicdo}{\setbox\fixbox\hbox{\ {} }\hskip-\wd\fixbox}
% end of hack
\newcommand{\algcost}[2]{\strut\hfill\makebox[1.5cm][l]{#1}\makebox[4cm][l]{#2}}



\title{Algorytmy i struktury danych -- notatki do kursu}
\author{mgr. inż Dominik Filipiak}
\date{Rok akademicki 2019/2020}

\begin{document}

\maketitle
\tableofcontents

\vspace{.25in}

Dokument ten jest pomocą dla prowadzącego i nie zastępuje w żaden sposób podręcznika akademickiego.
W szczególności nauka z tego dokumentu nie jest gwarantem zdania egzaminu.

\section{Wprowadzenie}

\subsection{O prowadzącym}
Mgr inż. Dominik Filipiak, Katedra Informatyki Ekonomicznej, p 3.24 CEUE (wejście przez 3.33), III piętro.
Strona WWW: \url{http://dfilipiak.com}, e-mail: \url{dominik.filipiak@standardo} (proszę zaczynać tytuły wiadomości od [ASD]).
Konsultacje (tylko po wcześniejszym umówieniu się mailowo, minimum 24h wcześniej - wiadomości wpadają często do spamu...): czwartki 16:00-17:00, piątki 12:00-13:00.

\subsection{Egzamin}
Egzamin z przedmiotu (wykład + ćwiczenia) odbędzie się:
\begin{itemize}
	\item I termin: na ostatnich ćwiczeniach 28 i 30 stycznia (w zależności od grupy na moich zajęciach),
	\item II termin: 12 lutego,
	\item III termin: 27 lutego.
\end{itemize}
Progi punktowe określa prof. Abramowicz.
Zaliczenie
Obowiązują ogólne zasady zaliczenia w KIE. Obecność na ćwiczeniach jest obowiązkowa. Każda nieobecność nieusprawiedliwiona począwszy od trzeciej włącznie to ujemne punkty na egzaminie (-3\%).
Zgodnie z regulaminem studiów, przy ponad połowie nieobecności (usprawiedliwionej bądź nie) jestem zmuszony przedstawić taką osobę dyrektorce studiów do skreślenia z listy studentów. Skany usprawiedliwień proszę przesyłać mailem w terminie zgodnych z zasadami zaliczania w KIE.

\subsection{Literatura}
\begin{itemize}
	\item Cormen Thomas H., Leiserson Charles E., Rivest Ronald L, Clifford Stein. \emph{Wprowadzenie do algorytmów}. Wydawnictwo Naukowe PWN, Warszawa, 2018.
	\item Lutz Mark. \emph{Python. Wprowadzenie}. Wydanie IV, Helion, 2011.
	\item Papadimitriou Christos H. \emph{Złożoność obliczeniowa}. Helion, 2012.
\end{itemize}
Większość użytych tu przykładów będzie pochodziła z książki Cormena i reszty.

\section{Proste algorytmy}

Definicja algorytmu, kod a pseudokod.

\subsection{Pierwiastki kwadratowe}
Niech $a,b,c \in \mathbb{R}, a \neq 0$.
Pierwiastki równania kwadratowego o postaci $y=ax+bx+c$ wyliczamy korzystając ze znanego ze szkoły średniej algorytmu.

\begin{algorithm}
    \caption{Pierwiastki rzeczywiste równania kwadratowego}
    \label{euclid}
    \begin{algorithmic}[1] % The number tells where the line numbering should start
        \Function{Quadratic-Roots}{$a, b, c$}
            \State $\Delta \gets b^2 - 4ac$
            \If{$\Delta > 0$} 
            	\State $x_1 \gets \frac{-b - \sqrt{\Delta}}{2a}$
            	\State $x_2 \gets \frac{-b + \sqrt{\Delta}}{2a}$
            \State \Return $\left \{x_1, x_2 \right \}$
            \ElsIf{$\Delta = 0$}
            		\State $x \gets \frac{-b}{2a} $
            		\State \Return $\{x \}$
            	\Else
            	\State \Return $\{\varnothing\}$
            \EndIf
        \EndFunction
    \end{algorithmic}
\end{algorithm}

\inputminted[linenos]{python}{code/2_quad.py}

\subsection{Algorytm Euklidesa}
Wprowadźmy najpierw operację dzielenia modulo.
\begin{equation*}
	a \bmod b = r \quad \Rightarrow \quad a = bn + r, \qquad |n| > r \ge 0
\end{equation*}

\begin{example}
\begin{align*}
7 \bmod 6 &= 1, \qquad (\text{ponieważ } 7 = 6 \cdot 1 + 1)\\
17 \bmod 7 &= 3, \qquad (\text{ponieważ } 17 = 7 \cdot 2 + 3)\\
-14 \bmod 2 &= 0\\
9 \bmod 6 &= 3\\
-17 \bmod 7 &= 4 
\end{align*}	
\end{example}


Według algorytmu Euklidesa, NWD($a,b$), gdzie $a, b \in $ wyliczymy w poniższy sposób: 
\begin{align*}
	a&=q_{1}b+r_{1}\\
	b&=q_{2}r_{1}+r_{2}\\
	r_{1}&=q_{3}r_{2}+r_{3}\\
	r_{2}&=q_{4}r_{3}+r_{4}\\
	&\,\,\,\vdots \\
	r_{n-2}&=q_{n}r_{n-1}+r_{n}\\
	r_{n-1}&=q_{n+1}r_{n}+0
\end{align*}
Jeżeli $r_{n+1}=0$, to NWD($a,b$) jest równe $r_{n}$.

\begin{example}
	Przykład: NWD dla 1071 oraz 462.
	\begin{align*}
		1071 &= 2 \cdot 462 + 147 \\
		462 &= 3 \cdot 147 + 21 \\
		147 &= 7 \cdot 21 + 0
	\end{align*}
	Wynikiem jest 21.
\end{example}


\begin{algorithm}
    \caption{Algorytm Euklidesa}
    \label{euclid}
    \begin{algorithmic}[1] % The number tells where the line numbering should start
        \Function{Euclid}{$a,b$}
            \State $r\gets a \bmod b$
            \While{$r\not=0$}
                \State $a \gets b$
                \State $b \gets r$
                \State $r \gets a \bmod b$
            \EndWhile
            \State \Return $b$
        \EndFunction
    \end{algorithmic}
\end{algorithm}

\inputminted[linenos]{python}{code/2_euclid_iterative.py}

\subsection{Silnia (iteracyjnie)}
Silnię definiujemy w następujący sposób:
\begin{equation}
	n!= 1 \cdot 2 \cdot \ldots \cdot n = \prod _{k=1}^{n}k\qquad {\mbox{dla }}n\in \mathbb{N}
\end{equation}
Już $11!$ to więcej, niż jest ludzi w Polsce.
\inputminted[linenos]{python}{code/2_factorial_iterative.py}

\subsection{Największy element ciągu}
Pomijalne. Wejście: lista liczb, wyjście: największy element z listy\footnote{Formalnie definicja jest nieco bardziej skomplikowana -- patrz \url{https://en.wikipedia.org/wiki/Maximal_and_minimal_elements}}.
\inputminted{python}{code/2_max.py}

\section{Listy i iteracje}

Tablica jest uporządkowaną kolekcją, w której każdy element ma swój indeks (dostęp bezpośredni po indeksie w stałym czasie).
Tablica jednowymiarowa jako wektor, dwuwymiarowa jako macierz, n-wymiarowa jako coś w rodzaju tensora.
Lista (jednokierunkowa) jest zbiorem elementów uporządkowanym liniowo (dostęp sekwencyjny, czas zależny od długości listy).
Niestety w języku Python pojęcia listy i tablicy są nieco pomieszane w stosunku do kanonu informatyki, tj. standardową strukturą danych jest coś na wzór ich hybrydy (o nazwie listy).
Różnicę między listą a tablicą dobrze widać w C++.

\subsection{Tworzenie i wydruk listy w Pythonie}
\begin{minted}{python}
list = [1, 2, 3, 4]

for i in list:
    print(i)

\end{minted}


\subsection{Tworzenie macierzy kwadratowej}
\begin{minted}{python}
X = [[12,7,3],
     [4 ,5,6],
     [7 ,8,9]]
     
a = [[1, 2, 3, 4], [5, 6], [7, 8, 9]]
for row in a:
    for elem in row:
        print(elem, end=' ')
    print()
\end{minted}


\subsection{Proste operacje na wektorze i macierzy}
Powiemy, że $C =AB$ dla macierzy $A$ o wymiarach $n \times m$ oraz macierzy $B$ o wymiarach $m \times p$, gdzie $c_{{ij}}=\sum _{{k=1}}^{m}a_{{ik}}b_{{kj}}$.
Złożoność obliczeniowa rzędu $\Theta(n^3)$ (dla macierzy $n \times n$, nieformalnie - bo trzy pętle) lub $\Theta(nmp)$ (dla macierzy $n \times m$ oraz $m \times p$)

\begin{algorithm}
    \caption{Mnożenie macierzy}
    \label{matmul}
    \begin{algorithmic}[1] % The number tells where the line numbering should start
        \Function{MatMul}{$A, B$}
            \State $C \gets $ nowa macierz o wymiarach $m \times p$
            	\For{$i$ from 1 to $n$}
            		\For{$j$ from 1 to $p$}
            			\State $sum \gets 0$
            			\For{$k$ from 1 to $m$}
            				\State $sum \gets sum + A_{ik}\cdot B_{kj}$
	             	\EndFor
	             	\State $C_{ij} \gets sum$
	            	\EndFor
            	\EndFor
            \State \Return $C$
        \EndFunction
    \end{algorithmic}
\end{algorithm}

\begin{minted}{python}
# source: https://www.programiz.com/python-programming/examples/multiply-matrix
# 3x3 matrix
X = [[12,7,3],
     [4 ,5,6],
     [7 ,8,9]]
# 3x4 matrix
Y = [[5,8,1,2],
     [6,7,3,0],
     [4,5,9,1]]
# result is 3x4
result = [[0,0,0,0],
          [0,0,0,0],
          [0,0,0,0]]

for i in range(len(X)):
   for j in range(len(Y[0])):
       for k in range(len(Y)):
           result[i][j] += X[i][k] * Y[k][j]

for r in result:
   print(r)
\end{minted}


\section{Rekurencja}

Podprogramy, rekurencja

\subsection{Silnia (rekurencyjnie)}
Niech $n \in \mathbb{N}$.

\begin{equation*}
n! = \begin{cases}
1 &\text{jeżeli $n =1 $}\\
n\cdot (n-1)! &\text{w każdym innym przypadku}
\end{cases}
\end{equation*}

\begin{example}

\begin{align*}
5! &= 5 \cdot 4! \\
&= 5 \cdot 4 \cdot 3! \\  
&= 5 \cdot 4 \cdot 3 \cdot 2! \\  
&= 5 \cdot 4 \cdot 3 \cdot 2 \cdot 1! \\
&= 5 \cdot 4 \cdot 3 \cdot 2 \cdot 1 \\
&= 120
\end{align*}

\end{example}


\begin{algorithm}
    \caption{Silnia (rekurencyjnie)}
    \label{factorial_recursive}
    \begin{algorithmic}[1] % The number tells where the line numbering should start
        \Function{Factorial}{$n$}
            	\If{$n = 1$}
            		\State \Return $n$
            	\Else
            		\State \Return $n \cdot$\Call{Factorial}{$n-1$}
            	\EndIf
        \EndFunction
    \end{algorithmic}
\end{algorithm}


\begin{minted}{python}
# source https://www.programiz.com/python-programming/examples/factorial-recursion
def factorial(n):
   if n == 1:
       return n
   else:
       return n*factorial(n-1)
\end{minted}

\subsection{Ciag Fibonacciego, min, max}
Ciąg Fibonacciego to ciąg, w którym każdy element począwszy od trzeciego jest sumą dwóch poprzednich elementów.
Niech $n \in \mathbb{N}$.
\begin{equation*}
F(n) = \begin{cases}
0 &\text{jeżeli $n =0 $}\\
1 &\text{jeżeli $n =1 $}\\
F_{n-1}+F_{n-2} &\text{w każdym innym przypadku}
\end{cases}
\end{equation*}

\begin{minted}{python}
def F(n):
    if n == 0: return 0
    elif n == 1: return 1
    else: return F(n-1)+F(n-2)
\end{minted}

\begin{example}
\begin{align*}
5! &= 5 \cdot 4! \\
&= 5 \cdot 4 \cdot 3! \\  
&= 5 \cdot 4 \cdot 3 \cdot 2! \\  
&= 5 \cdot 4 \cdot 3 \cdot 2 \cdot 1! \\
&= 5 \cdot 4 \cdot 3 \cdot 2 \cdot 1 \\
&= 120
\end{align*}	
\end{example}


Min i max w notebooku

\subsection{Algorytm Euklidesa (rekurencyjnie)}

\begin{minted}{python}
def gcd(a,b):
    if a % b == 0:
       return b
    return gcd(b, a % b)
\end{minted}

\section{Analiza algorytmów}

\subsection{Sortowanie przez scalanie}

Nieformalnie -- sortujemy jak talię kart, od lewej.

\begin{example}
Rozpatrzmy następującą tablicę. Zaczynamy od drugiego elementu:
$$\begin{bmatrix}
	5 & 2 & 4 & 6 & 1 & 3
\end{bmatrix}$$
Ponieważ $A[1] > A[0]$, to zamieniamy dwa pierwsze elementy miejscami i mamy:
$$\begin{bmatrix}
	2 & 5 & 4 & 6 & 1 & 3
\end{bmatrix}$$
Następnie rozpatrzmy trzeci element, czwórkę. Po przestawieniu mamy:
$$\begin{bmatrix}
	2 & 4 & 5 & 6 & 1 & 3
\end{bmatrix}$$
Szóstka jest w dobrym miejscu, nie zmienia się nic:
$$\begin{bmatrix}
	2 & 4 & 5 & 6 & 1 & 3
\end{bmatrix}$$
Jedynka wędruje na sam początek:
$$\begin{bmatrix}
	1 & 2 & 4 & 5 & 6 & 3
\end{bmatrix}$$
Pozostaje nam wziąć się za trójkę:
$$\begin{bmatrix}
	1 & 2 & 3 & 4 & 5 & 6
\end{bmatrix}$$
\end{example}

\begin{algorithm}
    \caption{Sortowanie przez scalanie (przykład z książki -- liczymy od 1!)}
    \label{insertion_sort}
    \begin{algorithmic}[1] % The number tells where the line numbering should start
        \Function{InsertionSort}{$A$}  \algcost{koszt}{krotność}
            	\For{$j \gets 2$ \textbf{to} $A$.$length $}  \algcost{$c_1$}{$n$}
            		\State $key \gets A[j]$ \algcost{$c_2$}{$n-1$}
            		\State // Wstaw $A[j]$ w posortowany ciąg $A[1 \ldots j-1]$ \algcost{$0$}{$n-1$}
            		\State $i \gets j - 1$ \algcost{$c_4$}{$n-1$}
            		\While{$i > 0$ \textbf{and} $A[i] > key$} \algcost{$c_5$}{$\sum_{j=2}^{n} t_{j} $}
            			\State $A[i+1] \gets A[i]$ \algcost{$c_6$}{$\sum_{j=2}^{n} \left ( t_{j} - 1\right )$}
            			\State $i \gets i - 1$ \algcost{$c_7$}{$\sum_{j=2}^{n} \left ( t_{j} - 1 \right )$}
            		\EndWhile
            		\State $A[i+1] \gets key$ \algcost{$c_8$}{$n - 1$}
            	\EndFor
        \EndFunction
    \end{algorithmic}
\end{algorithm}

\begin{minted}{python}
def insertionSort(A):
  print(A)
  for j in range(1, len(A)):
    key = A[j]
    i = j - 1
    while i >= 0 and A[i] > key:
      A[i+1] = A[i]
      i = i - 1
    A[i+1] = key
    print(A)
\end{minted}

Zbadajmy czas działania $T(n)$, gdzie $n$ jest długością wejściowej tablicy:
$$
T(n) = c_{1}n + c_2 (n-1) + c_4 (n-1) + c_5 \sum_{j=1}^{n} t_{j} + c_6 \sum_{j=2}^{n} \left ( t_{j} - 1\right )+ c_7 \sum_{j=2}^{n} \left ( t_{j} - 1\right ) + c_8 (n-1).
$$
W przypadku optymistycznym (gdy na wejściu dostajemy posortowaną tablicę) mamy $t_j = 1$ i tym samym omijamy linie 7 oraz 8 w algorytmie:
$$
\begin{bmatrix}
	1 & 2 & 3 & 4 & 5 & 6
\end{bmatrix}$$
\begin{align*}
T(n) &= c_{1}n + c_2 (n-1) + c_4 (n-1) + c_5 (n-1) + c_8 (n-1) \\
&= (c_1 + c_2 + c_4 + c_5 + c_8)n - (c_2 + c_4 + c_5 + c_8)
\end{align*}
Powyższe z kolei można przedstawić jako $an + b$ dla pewnych stałych $a$ oraz $b$, a więc jest to funkcja liniowa.

Co w przypadku pesymistycznym, gdy dostajemy \emph{najgorszą} tablicę do posortowania? Każdy element będzie musiał być przesuwany do końca, więc $t_j = j$.
$$
\begin{bmatrix}
	6 & 5 & 4 & 3 & 2 & 1
\end{bmatrix}$$
Wiedząc\footnote{Patrz szereg $1+2+3+4+\ldots$}, że:
$$\sum_{k=1}^n k = \frac{n(n+1)}{2}$$
mamy:
$$\sum_{j=2}^n j = \frac{n(n+1)}{2} - 1, \qquad \sum_{j=2}^n \left(j - 1 \right) = \frac{n(n+1)}{2}.$$
Tak więc w najgorszym wypadku:

\begin{align*}
T(n) &= c_{1}n + c_2 (n-1) + c_4 (n-1) + c_5 \left ( \frac{n(n+1)}{2} - 1 \right ) + c_6 \left ( \frac{n(n+1)}{2} \right ) + c_7 \left ( \frac{n(n+1)}{2} \right ) + c_8 (n-1)\\
&= \frac{1}{2} \left(c_5 + c_6 + c_7 \right) n^2 + \left(c_1 + c_2 +c_4 + \frac{1}{2} \left(c_5 - c_6 - c_7\right) + c^8 \right)n
\end{align*}
Powyższe można przedstawić jako $an^2 + bn + c$ dla pewnych stałych $a$, $b$ i $c$ -- jest to więc funkcja kwadratowa.

\subsection{Drzewa wywołań rekurencyjnych}
Cormen, s. 37

\section{Notacja asymptotyczna}

\subsection{Rzędy wielkości funkcji}


\section{Struktury danych}
\subsection{Kolejka}

\subsection{Stos}

\subsection{Drzewa}

\subsection{Drzewa BST}

\subsection{Kodowanie Huffmana}


\section{Algorytmy sortowania}

\subsection{Sortowanie bąbelkowe}

\subsection{Heapsort}

\subsection{Quicksort}

\section{Wstęp do teorii grafów i związane z nią algorytmy}
\subsection{Grafy oraz przechodzenie przez nie}
\subsection{Algorytm Dijkstry}
\subsection{Algorytm Prima}

\section{Wyszukiwanie wzorca w tekście}

\subsection{Wyszukiwanie naiwne}

\subsection{Algorytm Aho-Corasik}
\end{document}