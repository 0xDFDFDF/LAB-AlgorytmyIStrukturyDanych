\documentclass[10pt, oneside]{article} 
\usepackage{amsmath, amsthm, amssymb, calrsfs, wasysym, verbatim, bbm, color, graphics, geometry}
\usepackage{polski}
\usepackage[utf8]{inputenc}
\usepackage{minted}
\usepackage{algorithm}
\usepackage{algorithmicx}
\usepackage{algpseudocode}
\usepackage{url}

\geometry{tmargin=.75in, bmargin=.75in, lmargin=.75in, rmargin = .75in}  

\theoremstyle{remark}
\newtheorem*{example}{Przykład}


\title{Algorytmy i struktury danych -- notatki do kursu}
\author{mgr. inż Dominik Filipiak}
\date{Rok akademicki 2019/2020}

\begin{document}

\maketitle
\tableofcontents

\vspace{.25in}

Dokument ten jest pomocą dla prowadzącego i nie zastępuje w żaden sposób podręcznika akademickiego.
W szczególności nauka z tego dokumentu nie jest gwarantem zdania egzaminu.

\section{Wprowadzenie}

\subsection{Zasady zaliczenia przedmiotu}

\section{Proste algorytmy}

\subsection{Pierwiastki kwadratowe}
Niech $a,b,c \in \mathbb{R}, a \neq 0$.
Pierwiastki równania kwadratowego o postaci $y=ax+bx+c$ wyliczamy korzystając ze znanego ze szkoły średniej algorytmu.

\begin{algorithm}
    \caption{Pierwiastki rzeczywiste równania kwadratowego}
    \label{euclid}
    \begin{algorithmic}[1] % The number tells where the line numbering should start
        \Function{Quadratic-Roots}{$a, b, c$}
            \State $\Delta \gets b^2 - 4ac$
            \If{$\Delta > 0$} 
            	\State $x_1 \gets \frac{-b - \sqrt{\Delta}}{2a}$
            	\State $x_2 \gets \frac{-b + \sqrt{\Delta}}{2a}$
            \State \Return $\left \{x_1, x_2 \right \}$
            \ElsIf{$\Delta = 0$}
            		\State $x \gets \frac{-b}{2a} $
            		\State \Return $\{x \}$
            	\Else
            	\State \Return $\{\varnothing\}$
            \EndIf
        \EndFunction
    \end{algorithmic}
\end{algorithm}

\inputminted[linenos]{python}{code/2_quad.py}

\subsection{Algorytm Euklidesa}
Wprowadźmy najpierw operację dzielenia modulo.
\begin{equation*}
	a \bmod b = r \quad \Rightarrow \quad a = bn + r, \qquad |n| > |r| \ge 0
\end{equation*}

\begin{example}
\begin{align*}
7 \bmod 6 &= 1, \qquad (\text{ponieważ } 7 = 6 \cdot 1 + 1)\\
17 \bmod 7 &= 3, \qquad (\text{ponieważ } 17 = 7 \cdot 2 + 3)\\
-14 \bmod 2 &= 0\\
9 \bmod 6 &= 3\\
-17 \bmod 7 &= 4 
\end{align*}	
\end{example}


Według algorytmu Euklidesa, NWD($a,b$), gdzie $a, b \in $ wyliczymy w poniższy sposób: 
\begin{align*}
	a&=q_{0}b+r_{0}\\
	b&=q_{1}r_{0}+r_{1}\\
	r_{0}&=q_{2}r_{1}+r_{2}\\
	r_{1}&=q_{3}r_{2}+r_{3}\\
	&\,\,\,\vdots \\
	r_{n-1}&=q_{n-1}r_{n-2}+r_{n-1}\\
	r_{n}&=q_{n}r_{n-1}+r_{n}
\end{align*}
Jeżeli $r_n=0$, to NWD($a,b$) jest równe $r_{n-1}$.

\begin{example}
	Przykład: NWD dla 1071 oraz 462.
	\begin{align*}
		1071 &= 2 \cdot 462 + 147 \\
		462 &= 3 \cdot 147 + 21 \\
		147 &= 7 \cdot 21 + 0
	\end{align*}
	Wynikiem jest 21.
\end{example}


\begin{algorithm}
    \caption{Algorytm Euklidesa}
    \label{euclid}
    \begin{algorithmic}[1] % The number tells where the line numbering should start
        \Function{Euclid}{$a,b$}
            \State $r\gets a \bmod b$
            \While{$r\not=0$}
                \State $a \gets b$
                \State $b \gets r$
                \State $r \gets a \bmod b$
            \EndWhile
            \State \Return $b$
        \EndFunction
    \end{algorithmic}
\end{algorithm}

\inputminted[linenos]{python}{code/2_euclid_iterative.py}

\subsection{Silnia (iteracyjnie)}
Silnię definiujemy w następujący sposób:
\begin{equation}
	n!= 1 \cdot 2 \cdot \ldots \cdot n = \prod _{k=1}^{n}k\qquad {\mbox{dla }}n\in \mathbb{N}
\end{equation}
Już $11!$ to więcej, niż jest ludzi w Polsce.
\inputminted[linenos]{python}{code/2_factorial_iterative.py}

\subsection{Największy element ciągu}
Wejście: lista liczb, wyjście: największy element z listy\footnote{Formalnie definicja jest nieco bardziej skomplikowana -- patrz \url{https://en.wikipedia.org/wiki/Maximal_and_minimal_elements}}.
\inputminted{python}{code/2_max.py}

\section{Listy i iteracje}

\subsection{Tworzenie i wydruk listy}
\begin{minted}{python}
list = [1, 2, 3, 4]

for i in list:
    print(i)

\end{minted}


\subsection{Tworzenie macierzy kwadratowej}
\begin{minted}{python}
X = [[12,7,3],
     [4 ,5,6],
     [7 ,8,9]]
     
a = [[1, 2, 3, 4], [5, 6], [7, 8, 9]]
for row in a:
    for elem in row:
        print(elem, end=' ')
    print()
\end{minted}


\subsection{Proste operacje na wektorze i macierzy}
Powiemy, że $C =AB$ dla macierzy $A$ o wymiarach $n \times m$ oraz macierzy $B$ o wymiarach $n \times p$
$c_{{ij}}=\sum _{{k=1}}^{m}a_{{ik}}b_{{kj}}$, złożoność obliczeniowa rzędu $\Theta(n^3)$ (dla macierzy $n \times n$, nieformalnie - bo trzy pętle) lub $\Theta(nmp)$ (dla macierzy $n \times m$ oraz $m \times p$)

\begin{algorithm}
    \caption{Mnożenie macierzy}
    \label{euclid}
    \begin{algorithmic}[1] % The number tells where the line numbering should start
        \Function{MatMul}{$A, B$}
            \State $C \gets $ nowa macierz o wymiarach $m \times p$
            	\For{$i$ from 1 to $n$}
            		\For{$j$ from 1 to $p$}
            			\State $sum \gets 0$
            			\For{$k$ from 1 to $m$}
            				\State $sum \gets sum + A_{ik}\cdot B_{kj}$
	             	\EndFor
	             	\State $C_{ij} \gets sum$
	            	\EndFor
            	\EndFor
            \State \Return $C$
        \EndFunction
    \end{algorithmic}
\end{algorithm}

\begin{minted}{python}
# source: https://www.programiz.com/python-programming/examples/multiply-matrix
# 3x3 matrix
X = [[12,7,3],
     [4 ,5,6],
     [7 ,8,9]]
# 3x4 matrix
Y = [[5,8,1,2],
     [6,7,3,0],
     [4,5,9,1]]
# result is 3x4
result = [[0,0,0,0],
          [0,0,0,0],
          [0,0,0,0]]

for i in range(len(X)):
   for j in range(len(Y[0])):
       for k in range(len(Y)):
           result[i][j] += X[i][k] * Y[k][j]

for r in result:
   print(r)
\end{minted}


\section{Rekurencja}

Podprogramy, rekurencja


\subsection{Silnia rekurencyjnie}
${\displaystyle 1! = 1, \quad n!=n\cdot (n-1)!.}$

\begin{minted}{python}
# source https://www.programiz.com/python-programming/examples/factorial-recursion
def recur_factorial(n):
   if n == 1:
       return n
   else:
       return n*recur_factorial(n-1)
\end{minted}

\subsection{min, max, n elementów ciagu Fib.}
${\displaystyle F_{0}=0,\quad F_{1}=1, \quad F_{n}=F_{n-1}+F_{n-2},}$

\begin{minted}{python}
def F(n):
    if n == 0: return 0
    elif n == 1: return 1
    else: return F(n-1)+F(n-2)
\end{minted}

Min i max w notebooku

\subsection{Drzewa wywołań rekurencyjnych}
Cormen, s. 37

\end{document}